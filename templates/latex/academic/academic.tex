\documentclass[12pt]{article}

\usepackage[left=2.41cm, right=2cm, top=2.7cm, bottom=2cm]{geometry}
\usepackage[utf8]{inputenc}
\usepackage[ngerman]{babel}
\usepackage[ngerman]{translator}
\usepackage[T1]{fontenc}
\usepackage{lmodern}
\usepackage{enumitem}
\usepackage{textpos}
\usepackage{ragged2e}
\usepackage{multirow}
\usepackage{listings}
\usepackage{color}
\usepackage{xcolor}
\usepackage{mathtools}
\usepackage{graphicx}
\usepackage{url}
% euro symbol
\usepackage{eurosym}
% use procentual table column width
\usepackage{tabularx}
% use sans-serif font
\renewcommand*{\familydefault}{\sfdefault}
% load glossary and add it to table of references
\usepackage[toc,nonumberlist]{glossaries}
\makeglossaries
%  bibliografie with biblatex and biber
\usepackage[style=numeric, backend=biber, sortlocale=de]{biblatex}
\usepackage[autostyle]{csquotes}
\usepackage[nottoc]{tocbibind}
% line spacing
\usepackage{setspace}\onehalfspacing

\def \academictitle {Titel}
\def \academicauthor {Autor}

% create german date format
\usepackage{datetime}
\newdateformat{mytoday}{\twodigit{\THEDAY}.\twodigit{\THEMONTH}.\THEYEAR}

% cutom header and footer
\usepackage{fancyhdr}
\renewcommand{\headrulewidth}{0pt}
\renewcommand{\footrulewidth}{0pt}
\pagestyle{fancy}
\fancyhf{}
\fancyhead[RE,LO]{\nouppercase\leftmark}
\fancyhead[LE,RO]{\mytoday\today}
\fancyfoot[RE,LO]{\academicauthor}
\fancyfoot[LE,RO]{\thepage}
\renewcommand\sectionmark[1]{\markboth{#1}{}}

% custom caption for listings
\usepackage{caption}
\DeclareCaptionFont{white}{\color{white}}
\DeclareCaptionFormat{listing}{\colorbox{gray}{\parbox{\textwidth}{#1#2#3}}}
\captionsetup[lstlisting]{format=listing,labelfont=white,textfont=white}
\renewcommand\lstlistingname{Auflistung}
\renewcommand\lstlistlistingname{Auflistung}

% scala source code listing
\lstdefinelanguage{Scala}{
  morekeywords={abstract,case,catch,class,def,%
    do,else,extends,false,final,finally,%
    for,if,implicit,import,match,mixin,%
    new,null,object,override,package,%
    private,protected,requires,return,sealed,%
    super,this,throw,trait,true,try,%
    type,val,var,while,with,yield},
  sensitive=true,
  morecomment=[l]{//},
  morecomment=[n]{/*}{*/},
  morestring=[b]",
  morestring=[b]',
  morestring=[b]"""
}

\definecolor{intellijgreen}{RGB}{0,128,0}
\definecolor{intellijgray}{RGB}{128,128,128}
\definecolor{intellijblue}{RGB}{0,0,128}

% default settings for code listings
\lstset{
  frame=none,
  language=Scala,
  showstringspaces=false,
  columns=flexible,
  xleftmargin=20pt,
  basicstyle={\small\ttfamily},
  numbers=left,
  numbersep=5pt,
  numberstyle=\color{black},
  keywordstyle=\color{intellijblue},
  commentstyle=\color{intellijgray},
  stringstyle=\color{intellijgreen},
  breaklines=true,
  breakatwhitespace=true,
  tabsize=2
}

% pdf properties
\usepackage{hyperref}
\hypersetup{
    pdftitle={\academictitle},
    pdfauthor={\academicauthor},
    pdfsubject={\academictitle},
    pdfkeywords={\academictitle},
    hidelinks,
}
\newglossaryentry{wort}
{
  name=Wort,
  description={Erklärung}
}

\addbibresource{literature.bib}

\begin{document}

\renewcommand\refname{Quellenverzeichnis}

\title{\academictitle}
\author{\academicauthor}
\date{\today}

\begin{titlepage}
  \begin{center}
    \huge \textbf{\textsf{Wissenschaftliche Arbeit}} \\
    \vspace{2cm}
    \LARGE\textbf{\textsc{\academictitle}}\\
    \vspace{4cm}
    \large \textbf{Ort}\\
    \vspace{3cm}
  \end{center}
  \normalsize{
    \begin{tabular}{l p{1cm} l}
      \textbf{Name:} & & {\academicauthor} \\
      \textbf{Matrikelnummer:} & & {Nummer} \\
      \textbf{Fachbereich:} & & Fachbereich\\
      \textbf{Studiengang:} & & Studiengang\\
      \textbf{Fachsemester:} & & Semester\\
      \textbf{Professor:} & & Professor
    \end{tabular}\\
  }
\end{titlepage}

\tableofcontents
\newpage

\section{Einleitung}

Hier werden die beiden Quellen \cite{Ubuntu2015} und \cite{Fischer2014} für die Anzeige des Quellenverzeichnis zitiert.\\
\\
Hier wird der Eintrag \gls{wort} für die Anzeige des Glossars referenziert.\\
\\
Hier wird die Abbildung \ref{fig:ubuntu} für die Anzeige des Abbildungsverzeichnis referenziert. \\
\\
\begin{figure}[ht]
  \centering
  \includegraphics[width=\textwidth]{ubuntu.png}
  \caption{Ubuntu}
  \label{fig:ubuntu}
\end{figure}

\subsection{Motivation}

\subsection{Zielsetzung}

\newpage
\section{Grundlagen}

\newpage
\section{Analyse}

\subsection{Anforderungen}

\newpage
\section{Entwurf}

\newpage
\section{Implementierung}

\newpage
\section{Schlussbetrachtung}

\subsection{Ergebnis}

\subsection{Ausblick}

\subsection{Fazit}

\newpage
\printglossary[title=Glossar, toctitle=Glossar]

\newpage
\listoffigures

\newpage
\printbibliography[title=Quellenverzeichnis,heading=bibintoc]

\end{document}
